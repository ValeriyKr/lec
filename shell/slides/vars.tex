% vim: ft=tex cc=79 ts=3 sw=3 et
% 
% iframe = master frame; pframe = slave frame
\iframe{Переменные и параметры}

\begin{lstlisting}
name=value
name: [_A-Za-z][_A-Za-z0-9]
special name: * @ # ? - $ ! 0
\end{lstlisting}
\begin{table}[H]
\centering
\begin{tabular}{|c|c|c|}
\hline
\textbf{Имя} & \textbf{Значение} & \textbf{Экспортировать} \\
\hline name & value & \\
\hline name2 & value & + \\
\hline
\end{tabular}
\end{table}

Параметры, имя которых $\mathbb{N}$ называют позиционными: \lstinline{$1, $13, ...}
\begin{lstlisting}
$ ls -la /etc/passwd /etc/group
\end{lstlisting}

\end{frame}

\iframe{Специальные параметры}

\begin{table}[H]
\centering
\begin{tabular*}{\textwidth}{|l|l|}
\hline \textbf{Имя} & \textbf{Раскрывается в} \\
\hline
\lstinline{*} & позиционные параметры \\
\lstinline{@} & позиционные параметры \\
\lstinline{#} & число позиционных параметров \\
\lstinline{?} & код возврата последней команды \\
\lstinline{-} & опции интерпретатора \\
\lstinline{$} & PID интерпретатора \\
\lstinline{!} & PID последней <<свернутой>> команды \\
\lstinline{0} & argv[0] \\
\hline
\end{tabular*}
\end{table}

\end{frame}

\iframe{Специальные переменные}

\begin{table}[H]
\centering
\begin{tabular*}{\textwidth}{|l|l|}
\hline \textbf{Имя} & \textbf{Назначение} \\
\hline
EDITOR & Редактор \\
HOME & Путь к домашнему каталогу \\
IFS & Разделитель полей \\
PATH & Пути для поиска утилит \\
PPID & Идентификатор родительского \\
 & процесса \\
PS1 & Первое приглашение \\
PS2 & Последующие приглашения \\
PS3 & (ksh) Приглашение select \\
PS4 & Подсказка для отладки \\
\hline
\end{tabular*}
\end{table}

\end{frame}


\iframe{Подстановка значений}

\begin{enumerate}
\item \textit{Раскрытие тильды}, раскрытие параметров, подстановка результатов команды,
\textit{раскрытие арифметических операций}
\item Разделение на поля в зависимости от \lstinline{$IFS}
\item Подстановка путей к файлам (раскрытие глобов)
\item Удаление экранирующих символов
\end{enumerate}

\end{frame}

\iframe{Раскрытие тильды}

\begin{lstlisting}
$ echo ~
/home/valeriyk
$ echo ~root
/root
$ home=:~
$ echo $home
:/home/valeriyk
$ home=x~
$ echo $home
x~
\end{lstlisting}

\end{frame}

%TODO: fix
\iframe{Раскрытие параметров}

\begin{lstlisting}
${expression}

${parameter}

${parameter:-word}
${parameter:=word}
${parameter:?[word]}
${parameter:+word}

${parameter#pattern}
${parameter##pattern}
${parameter%pattern}
${parameter%%pattern}
\end{lstlisting}

\lstinline{:} -- параметр не установлен или пуст.

\end{frame}

\iframe{Переменные \$@ и \$*}
\begin{lstlisting}
$ set a b c
$ IFS=,
$ echo "$@"
a b c
$ echo "$*"
a,b,c
\end{lstlisting}

\end{frame}

\iframe{Раскрытие путей}

\begin{table}[H]
   \centering
   \begin{tabular*}{\textwidth}{|l|l|}
      \hline
      \texttt{*}      & любая строка, включая пустую \\
      \texttt{?}      & любой символ \\
      \texttt{[...]}  & любой из перечисленных символов \\
      \texttt{[!...]} & любой символ, кроме указанных \\
      \hline
   \end{tabular*}
\end{table}

\end{frame}
